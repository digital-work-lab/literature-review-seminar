\documentclass[11pt]{scrartcl}
%==============================
%ZUSÄTZLICH EINGEBUNDENE PAKETE
%==============================
\usepackage[T1]{fontenc}
\usepackage[utf8]{inputenc}
\usepackage[ngerman]{babel}
\usepackage{amsmath,amsthm,amssymb,euscript}
\usepackage{lmodern}
\usepackage{enumerate}
\usepackage{graphicx}
\usepackage{tabulary}
\usepackage{txfonts}
\usepackage{eurosym}
\usepackage{listings, color}
\definecolor{darkblue}{rgb}{0,0,.6}
\definecolor{darkred}{rgb}{.6,0,0}
\definecolor{darkgreen}{rgb}{0,.6,0}
\definecolor{red}{rgb}{.98,0,0}
\lstloadlanguages{Java}
\lstset{
basicstyle=\footnotesize\ttfamily,
showspaces=false,
showtabs=false,
columns=fixed,
numbers=left,
frame=none,
numberstyle=\tiny,
breaklines=true,
showstringspaces=false,
xleftmargin=1cm,
tabsize=4}

%tables
\usepackage{tabularx}

%circles in math mode
\usepackage{tikz}
\newcommand*\circled[1]{\tikz[baseline=(char.base)]{
            \node[shape=circle,draw,inner sep=2pt] (char) {#1};}}

%plotting
\usepackage{pgfplots}

\usepackage[a4paper]{geometry}
\geometry{a4paper,tmargin=3.5cm, bmargin=3cm, lmargin=3cm, rmargin=2cm, headheight=13em, headsep=3em, footskip=1cm}

\usepackage{textcomp}
\usepackage{longtable}
\usepackage{booktabs}

\usepackage{fancyhdr}
\setlength{\parskip}{1em}
\setlength{\parindent}{0pt}

\newcommand{\blatt}[1]{\begin{center}{\bf\Large  #1 \vspace*{0.3cm}}\end{center}}

\newtheoremstyle{aufgabenstyle}% name of the style to be used
  {24pt}% measure of space to leave above the theorem. E.g.: 3pt
  {6pt}% measure of space to leave below the theorem. E.g.: 3pt
  {\normalfont}% name of font to use in the body of the theorem
  {}% measure of space to indent
  {\bfseries}% name of head font
  {}% punctuation between head and body
  {\newline}% space after theorem head; " " = normal interword space
  {\thmname{#1}\thmnumber{ #2}\thmnote{ (#3)}\newline}% Manually specify head

\theoremstyle{aufgabenstyle}
\newtheorem{aufgabe}{\large Aufgabe}

%==============================
%BEGINN DER KOPF- UND FUSSZEILE
%==============================
\pagestyle{fancy}
\fancyhf{}
\fancyhead[R]{The Literature Review Seminar}
\fancyhead[L]{Literature Review Protocol Analysis Worksheet}
\setlength{\parskip}{1em}
\setlength{\parindent}{0pt}
\fancyfoot[C]{\thepage}
%\fancyfoot[L]{\today}
%================================
%BEGINN DES EIGENTLICHEN INHALTES
%================================

\begin{document}
	
\section*{Part I: Analyzing Exemplars}
%For each of the selected exemplar papers, identify the elements of the review protocol. Use references to specific sections/pages if applicable. Add notes or quotations where useful.

\subsection*{Papers: \_\_\_\_\_\_\_\_\_\_\_\_\_\_\_\_\_\_\_\_\_\_\_\_\_\_\_\_\_\_\_\_\_\_\_\_\_\_\_\_\_\_\_\_\_\_\_\_\_\_\_\_\_\_\_\_}

\renewcommand{\arraystretch}{1.9}

{\small
	% First half
	\begin{longtable}{|p{0.3\linewidth}|p{0.7\linewidth}|}
		\hline
		 & \textbf{Notes and Observations} (with references to sections/pages if applicable) \\
		\hline
		\textbf{Introduction} & \\
		Generativity Statement & \\[1.5em]
		Introduction of Topic / \newline  Phenomenon / Theory & \\[1.5em]
		Rationale for the Review & \\[1.5em]
		Review Objectives or Questions & \\[1.5em]
		\hline
		\textbf{Background} & \\
		Definition of Key Concepts & \\[1.5em]
		Framework / Theoretical \newline Foundations & \\[1.5em]
		\hline
		\textbf{Methods} & \\
		Type of Review and Justification & \\[1.5em]
		Search Strategy & \\[1.5em]
		Screening Procedures & \\[1.5em]
		Quality Appraisal Procedures & \\[1.5em]
		Data Extraction Procedures & \\[1.5em]
		Analysis Strategy and Procedures & \\[1.5em]
		\hline

	\end{longtable}
}

\vspace{1cm}

{\small
	% Second half
	\begin{longtable}{|p{0.3\linewidth}|p{0.7\linewidth}|}
		\hline
		 & \textbf{Notes and Observations} (with references to sections/pages if applicable) \\
		\hline

		\textbf{Main Sections} & \\
		Synthesis \newline ( Figure / Framework / Table) & \\[14em] \hline
		Research Agenda & \\[14em]
		\hline
		\textbf{Conclusions} & \\
		Main Contributions & \\[10em] \hline
		Implications \newline  (Research / Practice) & \\[10em]
		\hline
	\end{longtable}
}



\newpage

\section*{Part II: Planning Your Own Review}
Use this template to take notes on your own review protocol.

\renewcommand{\arraystretch}{1.9}

{\small
	% First half
	\begin{longtable}{|p{0.3\linewidth}|p{0.7\linewidth}|}
		\hline
		& \textbf{Notes and Observations} (with references to sections/pages if applicable) \\
		\hline
		\textbf{Introduction} & \\
		Generativity Statement & \\[1.5em]
		Introduction of Topic / \newline  Phenomenon / Theory & \\[1.5em]
		Rationale for the Review & \\[1.5em]
		Review Objectives or Questions & \\[1.5em]
		\hline
		\textbf{Background} & \\
		Definition of Key Concepts & \\[1.5em]
		Framework / Theoretical \newline Foundations & \\[1.5em]
		\hline
		\textbf{Methods} & \\
		Type of Review and Justification & \\[1.5em]
		Search Strategy & \\[1.5em]
		Screening Procedures & \\[1.5em]
		Quality Appraisal Procedures & \\[1.5em]
		Data Extraction Procedures & \\[1.5em]
		Analysis Strategy and Procedures & \\[1.5em]
		\hline
		
	\end{longtable}
}

\vspace{1cm}

{\small
	% Second half
	\begin{longtable}{|p{0.3\linewidth}|p{0.7\linewidth}|}
		\hline
		& \textbf{Notes and Observations} (with references to sections/pages if applicable) \\
		\hline
		
		\textbf{Main Sections} & \\
		Synthesis \newline ( Figure / Framework / Table) & \\[14em] \hline
		Research Agenda & \\[14em]
		\hline
		\textbf{Conclusions} & \\
		Main Contributions & \\[10em] \hline
		Implications \newline  (Research / Practice) & \\[10em]
		\hline
	\end{longtable}
}

\end{document}

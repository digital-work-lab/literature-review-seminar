\documentclass[11pt]{scrartcl}
%==============================
%ZUSÄTZLICH EINGEBUNDENE PAKETE
%==============================
\usepackage[T1]{fontenc}
\usepackage[utf8]{inputenc}
\usepackage[ngerman]{babel}
\usepackage{amsmath,amsthm,amssymb,euscript}
\usepackage{lmodern}
\usepackage{enumerate}
\usepackage{graphicx}
\usepackage{tabulary}
\usepackage{txfonts}
\usepackage{eurosym}
\usepackage{tabularx}   % For tabularx environment
\usepackage{booktabs}   % For \toprule, \midrule, \bottomrule
\usepackage{enumitem}   % For bullet points in cells
\usepackage{enumitem}
\usepackage{float}
\setlist{nosep}

\usepackage{listings, color}
\definecolor{darkblue}{rgb}{0,0,.6}
\definecolor{darkred}{rgb}{.6,0,0}
\definecolor{darkgreen}{rgb}{0,.6,0}
\definecolor{red}{rgb}{.98,0,0}
\lstloadlanguages{Java}
\lstset{
basicstyle=\footnotesize\ttfamily,
showspaces=false,
showtabs=false,
columns=fixed,
numbers=left,
frame=none,
numberstyle=\tiny,
breaklines=true,
showstringspaces=false,
xleftmargin=1cm,
tabsize=4}

%tables
\usepackage{tabularx}

%circles in math mode
\usepackage{tikz}
\newcommand*\circled[1]{\tikz[baseline=(char.base)]{
            \node[shape=circle,draw,inner sep=2pt] (char) {#1};}}

%plotting
\usepackage{pgfplots}

\usepackage[a4paper]{geometry}
\geometry{a4paper,tmargin=3.5cm, bmargin=3cm, lmargin=3cm, rmargin=2cm, headheight=13em, headsep=3em, footskip=1cm}

\usepackage{textcomp}
\usepackage{longtable}
\usepackage{booktabs}

\usepackage{fancyhdr}
\setlength{\parskip}{1em}
\setlength{\parindent}{0pt}

\newcommand{\blatt}[1]{\begin{center}{\bf\Large  #1 \vspace*{0.3cm}}\end{center}}

\newtheoremstyle{aufgabenstyle}% name of the style to be used
  {24pt}% measure of space to leave above the theorem. E.g.: 3pt
  {6pt}% measure of space to leave below the theorem. E.g.: 3pt
  {\normalfont}% name of font to use in the body of the theorem
  {}% measure of space to indent
  {\bfseries}% name of head font
  {}% punctuation between head and body
  {\newline}% space after theorem head; " " = normal interword space
  {\thmname{#1}\thmnumber{ #2}\thmnote{ (#3)}\newline}% Manually specify head

\theoremstyle{aufgabenstyle}
\newtheorem{aufgabe}{\large Aufgabe}

%==============================
%BEGINN DER KOPF- UND FUSSZEILE
%==============================
\pagestyle{fancy}
\fancyhf{}
\fancyhead[R]{The Literature Review Seminar}
\fancyhead[L]{Types of Literature Reviews}
\setlength{\parskip}{1em}
\setlength{\parindent}{0pt}
\fancyfoot[C]{\thepage}
%\fancyfoot[L]{\today}
%================================
%BEGINN DES EIGENTLICHEN INHALTES
%================================

\begin{document}

%\subsection*{Typology (Paré et al. 2015)}
%For each of the selected exemplar papers, identify the elements of the review protocol. Use references to specific sections/pages if applicable. Add notes or quotations where useful.

\begin{table}[H]
	\centering
	% \caption{Typology of literature review types (Grouped by Goal)}
	\footnotesize
	\begin{tabularx}{\textwidth}{l p{2.4cm} X X}
		\toprule
		\textbf{Goal} & \textbf{Review Type} & \textbf{Characteristics} & \\
		\midrule
		Describing & Narrative review & 
		\begin{itemize}[leftmargin=*, noitemsep, topsep=0pt]
			\item Scope: Broad
			\item Often without a methods section
			\item Search: Usually selective
			\item Sources: Conceptual and empirical
		\end{itemize} & 
		\begin{itemize}[leftmargin=*, noitemsep, topsep=0pt]
			\item Synthesis: Narrative summary informed by authorial experience, purposeful selection, and deliberate framing of the literature
		\end{itemize} \\
		Describing & Descriptive review & 
		\begin{itemize}[leftmargin=*, noitemsep, topsep=0pt]
			\item Scope: Broad
			\item Search: Representative
			\item Sources: Empirical and conceptual
			\item Explicit selection: Yes
		\end{itemize} & 
		\begin{itemize}[leftmargin=*, noitemsep, topsep=0pt]
			\item Synthesis: Descriptive analysis of meta-data and contents, often with plots or tables of distributions (e.g., journals, methods, levels of analysis)
		\end{itemize} \\
		\midrule
		Understanding & Scoping review & 
		\begin{itemize}[leftmargin=*, noitemsep, topsep=0pt]
			\item Scope: Broad
			\item Search: Comprehensive
			\item Sources: Conceptual and empirical
			\item Explicit selection: Yes
%			\item Quality appraisal: Not essential
		\end{itemize} & 
		\begin{itemize}[leftmargin=*, noitemsep, topsep=0pt]
			\item Synthesis: Content analysis to map emergent research areas and clarify the range, nature, and extent of research
		\end{itemize} \\
		Understanding & Critical review & 
		\begin{itemize}[leftmargin=*, noitemsep, topsep=0pt]
			\item Scope: Broad or narrow
			\item Search: Selective or representative
			\item Sources: Conceptual and empirical
			\item Explicit selection: Yes
%			\item Quality appraisal: Not essential
		\end{itemize} & 
		\begin{itemize}[leftmargin=*, noitemsep, topsep=0pt]
			\item Synthesis: Critical content analysis based on (pre-defined) criteria and constructive suggestions
		\end{itemize} \\
		\midrule
		Explaining & Theoretical review & 
		\begin{itemize}[leftmargin=*, noitemsep, topsep=0pt]
			\item Scope: Broad or narrow
			\item Search: Comprehensive
			\item Sources: Conceptual and empirical
			\item Explicit selection: Yes
			\item Quality appraisal: No
		\end{itemize} & 
		\begin{itemize}[leftmargin=*, noitemsep, topsep=0pt]
			\item Synthesis: Interpretive methods or inductive analysis (e.g., Grounded Theory), often combined with research propositions or a research agenda
		\end{itemize} \\
		Explaining & Realist review & 
		\begin{itemize}[leftmargin=*, noitemsep, topsep=0pt]
			\item Scope: Narrow
			\item Search: Iterative and purposive
			\item Sources: Conceptual and empirical
			\item Explicit selection: Yes
\item Quality appraisal: Yes
		\end{itemize} & 
		\begin{itemize}[leftmargin=*, noitemsep, topsep=0pt]
			\item Synthesis: Theory-driven and interpretive approaches (mixed methods), aimed to understand mechanisms
		\end{itemize} \\
		\midrule
		Testing & Qualitative systematic review & 
		\begin{itemize}[leftmargin=*, noitemsep, topsep=0pt]
			\item Scope: Narrow
			\item Search: Comprehensive
			\item Sources: Empirical (quant. or qual.)
			\item Explicit selection: Yes
			\item Quality appraisal: Yes
		\end{itemize} & 
		\begin{itemize}[leftmargin=*, noitemsep, topsep=0pt]
			\item Synthesis: Tabulating evidence according to pre-defined hypotheses or research models (vote counting)
		\end{itemize} \\
		Testing & Meta-analysis & 
		\begin{itemize}[leftmargin=*, noitemsep, topsep=0pt]
			\item Scope: Narrow
			\item Search: Comprehensive
			\item Sources: Empirical (quant. only)
			\item Explicit selection: Yes
			\item Quality appraisal: Yes
		\end{itemize} & 
		\begin{itemize}[leftmargin=*, noitemsep, topsep=0pt]
			\item Synthesis: Meta-analytic techniques (statistical methods), risk-of-bias reports, analyses of publication bias (forest plots), subgroup analyses
		\end{itemize} \\
		Testing & Umbrella review & 
		\begin{itemize}[leftmargin=*, noitemsep, topsep=0pt]
			\item Scope: Narrow or broad
			\item Search: Comprehensive
			\item Sources: Systematic reviews
			\item Explicit selection: Yes
			\item Quality appraisal: Yes
		\end{itemize} & 
		\begin{itemize}[leftmargin=*, noitemsep, topsep=0pt]
			\item Synthesis: Narrative synthesis, tabulating evidence aggregated by prior reviews
		\end{itemize} \\
		\bottomrule
	\end{tabularx}
\end{table}


\end{document}

\documentclass[11pt]{scrartcl}
%==============================
%ZUSÄTZLICH EINGEBUNDENE PAKETE
%==============================
\usepackage[T1]{fontenc}
\usepackage[utf8]{inputenc}
\usepackage[ngerman]{babel}
\usepackage{amsmath,amsthm,amssymb,euscript}
\usepackage{lmodern}
\usepackage{enumerate}
\usepackage{graphicx}
\usepackage{tabulary}
\usepackage{txfonts}
\usepackage{eurosym}
\usepackage{tabularx}   % For tabularx environment
\usepackage{booktabs}   % For \toprule, \midrule, \bottomrule
\usepackage{enumitem}   % For bullet points in cells
\usepackage{enumitem}
\usepackage{float}
\setlist{nosep}

\usepackage{listings, color}
\definecolor{darkblue}{rgb}{0,0,.6}
\definecolor{darkred}{rgb}{.6,0,0}
\definecolor{darkgreen}{rgb}{0,.6,0}
\definecolor{red}{rgb}{.98,0,0}
\lstloadlanguages{Java}
\lstset{
basicstyle=\footnotesize\ttfamily,
showspaces=false,
showtabs=false,
columns=fixed,
numbers=left,
frame=none,
numberstyle=\tiny,
breaklines=true,
showstringspaces=false,
xleftmargin=1cm,
tabsize=4}

%tables
\usepackage{tabularx}

%circles in math mode
\usepackage{tikz}
\newcommand*\circled[1]{\tikz[baseline=(char.base)]{
            \node[shape=circle,draw,inner sep=2pt] (char) {#1};}}

%plotting
\usepackage{pgfplots}

\usepackage[a4paper]{geometry}
\geometry{a4paper,tmargin=3.5cm, bmargin=3cm, lmargin=3cm, rmargin=2cm, headheight=13em, headsep=3em, footskip=1cm}

\usepackage{textcomp}
\usepackage{longtable}
\usepackage{booktabs}

\usepackage{fancyhdr}
\setlength{\parskip}{1em}
\setlength{\parindent}{0pt}

\newcommand{\blatt}[1]{\begin{center}{\bf\Large  #1 \vspace*{0.3cm}}\end{center}}

\newtheoremstyle{aufgabenstyle}% name of the style to be used
  {24pt}% measure of space to leave above the theorem. E.g.: 3pt
  {6pt}% measure of space to leave below the theorem. E.g.: 3pt
  {\normalfont}% name of font to use in the body of the theorem
  {}% measure of space to indent
  {\bfseries}% name of head font
  {}% punctuation between head and body
  {\newline}% space after theorem head; " " = normal interword space
  {\thmname{#1}\thmnumber{ #2}\thmnote{ (#3)}\newline}% Manually specify head

\theoremstyle{aufgabenstyle}
\newtheorem{aufgabe}{\large Aufgabe}

%==============================
%BEGINN DER KOPF- UND FUSSZEILE
%==============================
\pagestyle{fancy}
\fancyhf{}
\fancyhead[R]{Peer Review of Protocol Discussion Prompts}
\fancyhead[L]{Literature Review Seminar}
\setlength{\parskip}{1em}
\setlength{\parindent}{0pt}
\fancyfoot[C]{\thepage}
%\fancyfoot[L]{\today}
%================================
%BEGINN DES EIGENTLICHEN INHALTES
%================================

\begin{document}

%\subsection*{Typology (Paré et al. 2015)}
%For each of the selected exemplar papers, identify the elements of the review protocol. Use references to specific sections/pages if applicable. Add notes or quotations where useful.

\section*{Discussion Prompts: Peer Review of Protocols}

\textbf{TL;DR: You are the reviewer.}  
Ask your partner (the author) to briefly summarize her protocol.  
After the summary, \textbf{ask the questions below}—choose the most relevant ones or add your own.

\textbf{Reminder}: Be a \textbf{critical but constructive} reviewer.  
Maintain a professional and positive tone.

\paragraph{1. Review Type \& Goal Fit}
\begin{itemize}
	\item What is the specific topic, research question, and/or objective?
	\item What is your review type? Is it aimed at descriptive, understanding, explaining, testing?
	\item Why this type? How does it fit your research question?
	\item Is the review goal consistent with the chosen review type?
	\item How does the review type influence the methodological approach (e.g., inductive vs.\ deductive, qualitative vs.\ quantitative synthesis)?
\end{itemize}

\paragraph{2. Methodological Steps}
\begin{itemize}
	\item \textbf{Problem formulation}: How is the scope defined and justified?
	\item \textbf{Search strategy}: Which databases and techniques will be used? Why these?
	\item \textbf{Screening}: What are your inclusion and exclusion criteria? Are they appropriate?
	\item \textbf{Quality assessment}: Will you assess quality or bias? How?
	\item \textbf{Data extraction and analysis}:  
	Are you using inductive coding, vote counting, or meta-analysis?  
	Does the analysis align with the review type and research question?
	\item How will transparency and replicability be ensured (e.g., PRISMA, protocol registration)?
\end{itemize}

\paragraph{3. Contribution}
\begin{itemize}
	\item What type of contribution is expected (conceptual, theoretical, methodological, practical)?
	\item How will the results advance understanding (e.g., theory building, clarification of inconsistencies)?
	\item Are the contributions feasible given the scope and methods? Should the scope be adapted?
	% \item What makes the planned contribution \emph{novel} or \emph{impactful}?
	\item Did you find published reviews on the topic? If so, how does your work complement them?
\end{itemize}

\paragraph{4. Internal Coherence Check}
\begin{itemize}
	\item Do the goal, review type, methods, and contribution form a coherent whole?
	\item Would reviewers see a clear logic from research question to outcomes?
\end{itemize}

\subsection*{Reflection}
\begin{itemize}
	\item Strengths highlighted by your partner.
	\item Aspects to reconsider or clarify.
	\item Next steps for refining your protocol.
\end{itemize}

\end{document}
